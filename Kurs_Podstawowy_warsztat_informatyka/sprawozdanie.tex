\documentclass[a4paper]{article}
% Kodowanie latain 2
%\usepackage[latin2]{inputenc}
\usepackage[T1]{fontenc}
% Można też użyć UTF-8
\usepackage[utf8]{inputenc}

% Język
\usepackage[polish]{babel}
% \usepackage[english]{babel}

% Rózne przydatne paczki:
% - znaczki matematyczne
\usepackage{amsmath, amsfonts}
% - wcięcie na początku pierwszego akapitu
\usepackage{indentfirst}
% - komenda \url 
\usepackage{hyperref}
% - dołączanie obrazków
\usepackage{graphics}
% - szersza strona
\usepackage[nofoot,hdivide={2cm,*,2cm},vdivide={2cm,*,2cm}]{geometry}
\frenchspacing
% - brak numerów stron
\pagestyle{empty}

% dane autora
\author{Patryk Maciąg, 1014656432}
\title{Przykładowy plik w systemie \LaTeX}
\date{\today}

% początek dokumentu
\begin{document}
\maketitle
\section{uname -a}
Skrót od "unix name". Jak nazwa podpowiada, polecenie uname wypisuje informacje o systemie operacyjnym. Samo uname wypisze jedynie nazwę jądra, czyli jeśli korzydtając z linuxa, otrzymasz nazwę Linux. Polecenie uname można teżużywać z różnymi flagami, żeby otrzymać więcej informacji o systremie. Wystarczy użyć flagi -a, żeby otrzymać wszytskie dostępne informacje, takie jak: nazwa jądra i hosta, wydanie jądra, wersja jądra, nazwa sprzętu, nazwa systemu operacyjnego.

\section{Czas uniksowy}
Zwany też POSIX - rzeczywista liczba sekund, jakie upłynęły od "początku epoki Uniksa", czyli początku roku 1970 (UTC). 
\section{Problem roku 2038}
Jest to problem zliczania czasu przez oprogramowania korzystające z czasu uniksowego. Mianowicie dotyczy on osiągnięcia przez licznik sekund swojej maksymalnej wartości, która wynosi \(2147483647\) (ponieważ licznik sekund jest reprezentowany przez 32-bitową liczbę sekund ze znakiem, czyli wartość maksymalna wynosi \(2^{31}-1\)). Wtedy nastąpi powrót wartości licznika do 0, czyli początku 1970, lub nastąpi przeskok do najmniejszej wartości ujemnej.
\section{Różnice plików}
Jedyne różnice jakie zauważyłem to pojawienie się zakładek.
\section{Wzór z Whitebook'a}
\(\textbf{Zadanie 227. } \bigcup_{n=0}^{\infty}\bigcap_{k=n}^{\infty} A_k\subseteq\bigcap_{n=0}^{\infty}\bigcup_{k=n}^{\infty} A_k\)
\end{document}
